\documentclass{article}
\usepackage{amsmath}
\usepackage{amsfonts}
\usepackage{amssymb}
\usepackage{sympytex}

\usepackage{pgfplots}

\title{SympyTeX pgfplots example}
\author{Alexander Steppke}

\begin{document}

\maketitle

\section{Plotting with PGF/TikZ in SympyTeX}
To plot figures with SympyTeX besides the python matplotlib alternatively the LaTeX package pgfplots can be used. Pgfplots is using the PGF/TikZ package as a graphics backend. The main advantages are the fast output, as we do not need to create any external files and the good integration of pgfplots into LaTeX. A short example is given below.

\begin{sympyblock}
from numpy import linspace, sin
time = linspace(0,6,200)
values = sin(3*time)
coordinates = ""
for point in zip(time, values):
  coordinates = coordinates + str(point)

pgfplot = r"""
\begin{tikzpicture}
  \begin{axis}[xlabel=Time,ylabel=Amplitude]
    \addplot[color=red,mark=x] coordinates { %s };
  \end{axis}
\end{tikzpicture}
""" % (coordinates)

\end{sympyblock}

After calculating the datapoints we insert them into the string {\verb pgfplot } and output the string into our LaTeX document using {\verb \sympyplain{pgfplot} } in a figure environment.
\begin{figure}
  \centering
  \sympyplain{pgfplot}
  \caption{A simple plot with SympyTeX and pgfplots.}
\end{figure}


\end{document}
